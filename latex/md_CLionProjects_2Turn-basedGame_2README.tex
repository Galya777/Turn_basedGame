\chapter{README}
\hypertarget{md_CLionProjects_2Turn-basedGame_2README}{}\label{md_CLionProjects_2Turn-basedGame_2README}\index{README@{README}}
Походова игра

Кралят Лич се е завърнал. Той иска да унищожи всичко живо в света и няма да се спре пред нищо, докато не го направи. От вас се иска да го спрете, като реализирате симулация на походова игра със следните особености и правила.

В света съществуват два вида единици -\/ живи и немъртви. Живите единици са хора с различни способности. Немъртвите единици са същества, които са били чисти немъртви създания или въздигнати немъртви създания. Всички единици имат здраве и броня от различен тип, с определена здравина, както и атака, която нанасят, и злато, което струват, за да бъдат създадени. Някои единици притежават и манна. Различните видове живи и немъртви единици са описани в следващите параграфи. Помислете как да бъдат организирани в класове. Целта е лесно да се добавят нови видове.

Пехотинец -\/ основната част от живите единици. Те биват създадени със следните характеристики\+: 420 здраве, Medium Type броня със здравина 8, атака със стойност 7, струват 250 злато. Стрелец -\/ жива единица, снабдена с лък или огнестрелно оръжие. Те биват със следните характеристики\+: 535 здраве, Medium Type броня със здравина на бронята 3, атака със стойност 10, струват 300 злато. Рицар -\/ жива единица, тежковъоръжена и добре защитена. Те биват създадени със следните характеристики\+: 835 здраве, Heavy Type броня със здравина на бронята 10, атака със стойност 45, струват 700 злато. Лечител -\/ жива единица, която може да възстановява здравето на други живи единици. Лечителите имат здраве и манна. Създават се със следните характеристики\+: 290 здраве, 200 манна, Unarmored Type броня със здравина на бронята 0, атака със стойност 2, струват 150 злато. Възстановява 100 здраве на друга жива единица и това струва 100 манна на лечителя. Магьосник -\/ жива единица, която може да създава заклинания. Тези единици също имат и манна. Те биват създадени със следните характеристики\+: 325 здраве, 200 манна, Leather Type броня със здравина на бронята 3, атака със стойност 10, струват 250 злато. Атаката на магьосника струва 50 манна.

Скелет -\/ основната сила на немъртвите. Те биват създадени с характеристики\+: 500 здраве, Medium Type броня със здравина на бронята 8, атака със стойност 5, струват 100 злато. Гул -\/ представител на немъртвите единици. Създават се със следните характеристики\+: 400 здраве, Heavy Type броня със здравина на бронята 6, атака със стойност 12, струват 250 злато. Некромант -\/ немъртва единица, която може да извършва заклинания. Създават се с характеристики\+: 300 здраве, 200 манна, Unarmored Type броня със здравина на бронята 0, атака със стойност 4, струват 400 злато. Некромантите могат да призовават до 3 скелета в рамките на ход, ако съществуват 3 мъртви единици от страна на живите от предишен ход. Това им коства 150 манна. Зомби -\/ немъртва единица. Те биват създадени със следните характеристики\+: 250 здраве, Unarmored Type броня със здравина на бронята 0, атака със стойност 15, струват 300 злато. Дибук -\/ вид зомби. Притежава всички характеристики на зомбитата, но има и манна. Манната на дибук е 300. Останалите стойности за характеристики повтарят тези при зомбитата. Атаката му струва 150 манна. Ревенант -\/ вид зомби. Притежава характеристиките на зомбитата, но биват променени следните\+: 600 здраве, Unarmored Type броня със здравина на бронята 0. Призрак -\/ особен вид немъртво създание. Това създание няма нито здраве, нито манна, няма броня и няма атака. Особеното при този вид немъртво създание е, че има възможността в рамките на един ход да добави 250 здраве към произволна немъртва единица, на цената на това да се самоунищожи. Струва 500 злато.

Всяка от двете фракции има и главнокомандващи, които се различават с различни способности един от друг. Играта не забранява в дадена фракция да има повече от един вид главнокомандващ от вид\+:

Лич -\/ могъщо немъртво създание. Притежава здраве и манна. Личовете имат следните характеристики\+: 1500 здраве, 1000 манна, Heavy Type броня със здравина на бронята 15, атака със стойност 100. Основното му заклинание е да въздига мъртъвци. Въздигането на мъртъвци се осъществява, ако по време на поредния ход на играта има живи създания, които са загубили дуела в рамките на хода. В случай че в същия този ход са участвали повече от един лич от страна на немъртвите, то въздигането на мъртъвци се осъществява точно веднъж. Въздигането на мъртъвци струва 750 манна на един лич. Повелител на ужаса -\/ основните причинители за немъртвата напаст. Притежава здраве и манна. Повелителите на ужаса имат следните характеристики\+: 3000 здраве, 2000 манна, Heavy Type броня със здравина на бронята 20, атака със стойност 200. Основното му заклинание е да призовава некроманти и гулове. Призоваването на немъртви създания изисква съответно 400 манна за призоваване на некромант и 500 манна за призоваване на гул. Мъртъв рицар -\/ паднал рицар. Това е специален вид немъртъв, който може да бъде създаден единствено след смъртта на жив рицар. Мъртвият рицар притежава здраве и манна. Неговите характеристики са, както следва\+: 2500 здраве, 1000 манна, Heavy Type броня със здравина на бронята 15, атака със стойност 150. Тези немъртви могат да възвръщат здраве на все още съществуващи в битката мъртви единици, като всяко използване на такава способност струва 350 манна от страна на мъртвия рицар. Създаването на такъв главнокомандващ се случва с 25 \% шанс след смъртта на жив рицар. Ловец на немъртви -\/ могъщ човешки воин. Притежава здраве и манна. Ловците на немъртви имат следните характеристики\+: 2000 здраве, 1500 манна, Heavy Type броня със здравина на бронята 17, атака със стойност 75. Основното му заклинание е да унищожава главнокомандващи на немъртвите. С други думи, ако в рамките на един ход в групата на немъртвите участва главнокомандващ от немъртвите, а в групата на живите единици участва ловец на немъртви, то той унищожава мигновено точно един от главнокомандващите на немъртвите. Това му коства 1000 манна. Танцуващ с остриета -\/ основна сила на хората. Танцуващите с остриета имат следните характеристики\+: 4000 здраве, Medium Type броня със здравина на бронята 25, атака със стойност 300. Няма способности. Паладин -\/ човешки воин, отдаден на силата на Светлината. Притежава здраве и манна. Паладините имат следните характеристики\+: 5000 здраве, 3000 манна, Heavy Type броня със здравина на бронята 20, атака със стойност 250. Той има специалната сила да възкресява победени живи единици отново като живи бойци. Негова сила е да възвърне изгубено здраве на все още жива единица, което изразходва 500 манна от паладина.

Правила за здравето\+: Здравето на живи единици може да се възстанови единствено от живи единици, които са лечители или главнокомандващи живите единици с необходимите характеристики. Здравето на немъртви единици може да се възстанови единствено от некромантите или главнокомандващи мъртвите единици с необходимите характеристики. След като се извърши ходът на живите или на немъртвите единици, другата страна в дуела автоматично увеличава 100 единици от здравето на всяка своя единица.

Правила за манната\+: Манната на живите единици и манната на немъртвите единици е еквивалентна характеристика. След като се извърши ходът на живите или на немъртвите единици, другата страна в дуела автоматично увеличава с 75 единици манната на всяка своя единица.

Правила за бронята\+: Бронята позволява част от нанесен удар да бъде ограничена. Различните типове броня имат различен процент на намаляване на противников удар. Типовете Unarmored Type, Leather Type, Medium Type, Heavy Type брони съответно спират 0\%, 25\%, 50\%, 75\% от нанесения удар. Стойността на бронята показва колко пъти съответната единица може да използва бронята си. Когато я използва веднъж, нейната стойност се намалява с 1 за съответната единица. Когато стигне стойност 0, единицата няма право да използва бронята си повече.

Правила за единиците, които не са главнокомандващи\+: Всяка от страните започва със своя база. Базата представлява своеобразна абстракция, която може да генерира единици. Във всяка игра има точно 2 бази. Генерирането на единици става със злато, определено количество от което всяка страна притежава в началото на играта -\/ задава се от конфигурационен файл. Броят на текущо генерирани единици в рамките на играта не може да надвишава определен лимит. Този лимит също се задава от конфигурационен файл. Генерираните единици не включват главнокомандващите.

Не се иска от реализацията на проекта създаването на уникални немъртви въздигнати единици. С други думи, ако някоя жива единица е вече мъртва в рамките на дуел, например Лечител, то той може да бъде въздигнат според посочените по-\/горе правила, например Скелет. В противен случай това нарушава логиката на играта и би довело до несмислена двузначност за много от единиците.

Правила за главнокомандващите\+: Всеки главнокомандващ се характеризира с уникално име, освен с посочените характеристики. Главнокомандващите на всяка от страните се четат от файл и, веднъж прочетени, не могат да бъдат премахнати от играта, докато не бъдат унищожени в контекста на самата игра като единици. Не е позволено да съществуват повече от 7 единици главнокомандващи от един вид, във всяка от страните на живите или на немъртвите. Параметрите за главнокомандващите трябва да бъдат прочетени от конфигурационен файл, с формат по ваше усмотрение, в началото на играта.

Правила за дуел\+: Първият играч в играта е винаги с живите единици, а симулаторът е вторият играч, като той симулира поведението на немъртвите единици. Вторият играч -\/ симулаторът, избира на случаен принцип колко от своите единици да включи в текущия дуел, като избира на случаен принцип и кои генерали да участват в дуела. Имате свобода са изберете как да генерирате немъртви единици, които не са главнокомандващи. След като потребителят се запознае с тази информация, той има правото да\+:

Създаде единица с командата CREATE, посочвайки нейното име. Избере главнокомандващ за дуела с командата SELECT BOSS и име. Избере брой обикновени единици за дуела с командата SELECT  $<$\+TIMES$>$ и име на единицата, където $<$\+TIMES$>$ е броят на единиците от конкретен вид, който иска да бъде включен. Да участва в дуела с командата START.

След това дуелът започва с атакуване от страна на немъртвите. Главнокомандващите имат право да използват своите способности, ако е допустимо. Следва контраатака от страна на живите. Те имат право да използват своите способности, ако е допустимо. Следва въздигане на живи единици от страна на немъртвите, ако е възможно. Следва последна атака от страна на живите. Преценете как най-\/удобно да реализирате самата динамика на дуела.

След приключване на дуел, всяка от фракциите получава по 1000 злато. Симулацията се печели от фракцията, която спечели първа 3 дуела. Печелене на дуел за A означава, при сблъсък на A и B, фракцията B не трябва да има никакви единици. Единиците, останали след последните атаки и използване на възможности, могат да участват повторно в следващи дуели, с актуализирани стойности за здраве и манна, ако е наложителна актуализация.

Във всеки един момент потребителят трябва да може да запише текущото състояние на играта (Save) или да я прекрати (Exit). При прекратяване програмата трябва да извежда запитване към играча дали иска да запише или не текущото състояние. Трябва да може също да се зареди предварително запазено състояние (Load), както и да се поддържат възможности за рестартиране от началото на играта (Restart). При запазване във файл се запазват следните данни\+: златото на двете фракции, броят спечелени дуели на всяка от фракциите, главнокомандващите единици, останали за използване -\/ техните имена и характеристики. Не се записват във файл данните за единиците, които не са главнокомандващи, дори да съществуват такива.

Помислете как да визуализирате текущото състояние и да организирате взаимодействието на играча със системата. Примерен начин е това да се направи чрез опционално меню.

Справочник за хората, незапознати с термините в подобен род игри\+: здраве (health), манна (mana), броня (armor).

Всяка едно разширение на играта, което смятате за смислено и което не нарушава посочените правила, може да реализирате по свой начин, стига да е смислен. Условието предполага известна свобода и, стига да са изпълнени посочените изисквания в условието, може да бъде реализирана по избран от програмиста начин. Вие може да изберете валиден формат на данните, менюто за интеракция с потребителя… 